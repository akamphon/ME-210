\setcounter{secnumdepth}{1}
\setcounter{tocdepth}{1}

\usepackage[utf8]{inputenc} % set input encoding (not needed with XeLaTeX)

%%% PAGE DIMENSIONS
\usepackage{geometry} % to change the page dimensions
\geometry{a4paper} % or letterpaper (US) or a5paper or....
\geometry{margin=1in} % for example, change the margins to 2 inches all round
% \geometry{landscape} % set up the page for landscape

\usepackage{graphicx} % support the \includegraphics command and options

\usepackage[parfill]{parskip} % Activate to begin paragraphs with an empty line rather than an indent

%%% PACKAGES
\usepackage{booktabs} % for much better looking tables
\usepackage{array} % for better arrays (eg matrices) in maths
\usepackage{enumitem}
\usepackage{verbatim} % adds environment for commenting out blocks of text & for better verbatim
\usepackage{subfig} % make it possible to include more than one captioned figure/table in a single float

%%% HEADERS & FOOTERS
\usepackage{fancyhdr} % This should be set AFTER setting up the page geometry
\pagestyle{fancy} % options: empty , plain , fancy
\renewcommand{\headrulewidth}{1pt} % customise the layout...
\renewcommand{\footrulewidth}{1pt}
\lhead{\rightmark}\chead{}\rhead{\thepage}
\lfoot{ME 210: Mechanical of Materials}\cfoot{}\rfoot{S. Akamphon}

\usepackage[titles,subfigure]{tocloft} % Alter the style of the Table of Contents

\usepackage{newtxtext}
\usepackage{newtxmath}
\usepackage{hyperref}
\hypersetup{
  colorlinks,
  citecolor=black,
  filecolor=black,
  linkcolor=black,
  urlcolor=black
}
\let\openbox\relax
\usepackage{amsmath}
\usepackage{amsthm}
\usepackage{amssymb}
\usepackage{multirow}
\usepackage{gensymb}
\usepackage{array}
\usepackage{cleveref}
\usepackage{siunitx}

%% Chapter Heading %%%%%
\usepackage[explicit]{titlesec}
\newcommand*\chapterlabel{}
%\usepackage{kpfonts}
%% Drawing pics with tikz %%%
\usepackage{tikz,calc}
\tikzset{>=latex}
\usetikzlibrary{arrows,calc,decorations,shapes,decorations.pathmorphing,patterns,snakes,decorations.fractals,shadings,lindenmayersystems,shadows}
\newcommand{\AxisRotator}[1][rotate=0]{%
    \tikz[x=0.25cm,y=0.60cm,line width=.2ex,-stealth,#1] \draw(0,0) arc (150:-150:1 and 1);%
}

\titleformat{\chapter}
{\gdef\chapterlabel{}
  \normalfont\huge\bfseries}
{\gdef\chapterlabel{\thechapter}}{0pt}
{\begin{tikzpicture}[remember picture,overlay]
    \node[yshift=-3cm] at (current page.north west)
    {\begin{tikzpicture}[remember picture,overlay]
        \draw[fill=LightSkyBlue] (0,0) rectangle
        (\paperwidth,3cm) ;
        \node[anchor=east,xshift=.9\paperwidth,rectangle,
        rounded corners=5pt,inner sep=11pt,
        fill=DarkBlue]
        {\color{white}#1};
        \ifnum\value{chapter}>0
        \node[right, xshift=1cm, yshift=1.5cm]{\scalebox{4}{\Huge\thechapter}};
        \fi
      \end{tikzpicture}   
    };
  \end{tikzpicture}
}

\titleformat{name=\chapter, numberless}
{\gdef\chapterlabel{}
  \normalfont\huge\bfseries}
{\gdef\chapterlabel{\thechapter}}{0pt}
{\begin{tikzpicture}[remember picture,overlay]
    \node[yshift=-3cm] at (current page.north west)
    {\begin{tikzpicture}[remember picture,overlay]
        \draw[fill=LightSkyBlue] (0,0) rectangle
        (\paperwidth,3cm) ;
        \node[anchor=east,xshift=.9\paperwidth,rectangle,
        rounded corners=5pt,inner sep=11pt,
        fill=DarkBlue]
        {\color{white}#1};
      \end{tikzpicture}   
    };
  \end{tikzpicture}
}

\titlespacing*{\chapter}{0pt}{50pt}{-10pt}

%%% New column type
\newcolumntype{L}[1]{>{\raggedright\let\newline\\\arraybackslash\hspace{0pt}}m{#1}}
\newcolumntype{C}[1]{>{\centering\let\newline\\\arraybackslash\hspace{0pt}}m{#1}}
\newcolumntype{R}[1]{>{\raggedleft\let\newline\\\arraybackslash\hspace{0pt}}m{#1}}

%Example environment
\usepackage{thmtools}
\usepackage{mdframed}
\usepackage{float}
\definecolor{example}{RGB}{204,255,204}
\definecolor{titlepagecolor}{cmyk}{0.3,.50,0.50,.40}
\definecolor{namecolor}{cmyk}{0.3,.50,0.5,.10}
\declaretheoremstyle[
spaceabove=6pt, spacebelow=0pt,
headfont=\normalfont\bfseries,
notefont=\mdseries, notebraces={(}{)},
bodyfont=\normalfont,
postheadspace=1em,
numberwithin=chapter,
preheadhook={\begin{mdframed}[backgroundcolor=LightSkyBlue,
    innertopmargin=6pt, splittopskip=\topskip, % 
    skipbelow= 0pt, skipabove=6pt, %
    topline=false,bottomline=false,leftline=false,rightline=false] \sloppy},
  postfoothook=\end{mdframed},
headpunct={}
]{exstyle}
\declaretheoremstyle[
spaceabove=6pt, spacebelow=6pt,
headfont=\normalfont\bfseries,
notefont=\mdseries, notebraces={(}{)},
bodyfont=\normalfont,
postheadspace=1em,
preheadhook={\begin{mdframed}[backgroundcolor=LightSkyBlue,
    innertopmargin =6pt, splittopskip = \topskip, % 
    skipbelow=6pt, skipabove=0pt, %
    topline=false,bottomline=false,leftline=false,rightline=false] \sloppy},
  postfoothook=\end{mdframed},
headpunct={},
numbered=no
]{solstyle}
\declaretheorem[style=exstyle]{example}
\declaretheorem[style=solstyle]{solution}

% New list for exercises
\newlist{exercises}{enumerate}{2}
\setlist[exercises]{%
  label=\textbf{\thechapter-\arabic*}~,  % Label: Exercise C.E
  ref=\thechapter-\arabic*, % References: C.E (important!)
  align=left,               % Left align labels
  labelindent=0pt,          % No space betw. margin of list and label
  leftmargin=0pt,           % No space betw. margin of list and following lines
  itemindent=!,             % Indention of item computet automatically
}

%\setlist[enumerate, 1]{label=(\alph*)}            % Label for subexercises

\newcommand{\exercise}{%
\item\label{lab:\arabic{chapter}.\arabic{exercisesi}}  % Append label to item
}

%% bibliography %%%%

\usepackage[style=numeric,backend=biber]{biblatex}
\addbibresource{me210.bib}

%%% END Article customizations